\documentclass[12pt]{article}

\usepackage[margin=1cm, paperheight=44in]{geometry}
\usepackage{amsfonts, multicol, amsxtra}

\title{Function Homework}
\author{Michael Padilla}

\begin{document} 
\maketitle
\section*{Exercises for Section 17.1}
\begin{enumerate}
    \item domain = \{0,1,2, 3, 4\}, range = \{2,3,4\}, f(2) = 4, f(1) = 3
    \item domain = A, range = {2,3,4,5}, f(b) = 3, f(d) = 5
    \item  \{(a,0), (b, 0)\}, \{(a, 0), (b,1)\}, \{(a,1), (b,1)\}, \{(a,1), (b,0)\}
    \item \{(a,0), (b, 0), (c,0)\}, \{(a, 0), (b,1), (c, 0)\}, \{(a,0), (b,1), (c,1)\}, \{(a,0), (b,0), (c,0)\}, \{(a,1), (b,0), (c, 0)\}, \{(a,1), (b, 1), (c,0)\}, \{(a, 1), (b, 0), (c,1)\}, \{(a, 1), (b,1), (c,1)\} 
    \item  \{(a, d)\}
\end{enumerate}
\section*{Exercises for Section 17.2}
\begin{enumerate}
    \item \{(1,a), (2,b), (3,b), (4,a)\}
	\item 
	    \begin{equation*}
	    	\begin{split}
		    f(a) - f(b) \neq 0\\
		    ln(a) - ln(b)\\
		    e^{ln(a)} - e^{ln(b)}\\
		    a -b \neq 0\\
		    \text{Therefore, it's Injective}\\
		    f(a) = b\\
		    ln(a) = b\\
		    e^b = a\\
		    b \in \mathbb{Z}\\
		    \text{Therefore, it's Surjective}
	    	\end{split}
	    \end{equation*}
	\item [5]
	    \begin{equation*}
	    	\begin{split}
		    f(a) - f(b) \ne 0\\
		    2a+1 - (2b + 1)\\
		    = 2a + 1 -2b -1\\
		    = 2a-2b \ne 0\\
		    \text{Therefore, it's Injective}\\
		    f(a) = b,\ odd\\
		    2a+1 = b\\
		    2a = b-1\\
		    a = \dfrac{b-1}{2} \notin \mathbb{Z}\\
		    \text{Therefore, it's not Surjective}
	    	\end{split}
	    \end{equation*}
	\item [6]
	    \begin{equation*}
	    	\begin{split}
		    f(a1, b1) -f(a2, b2) \ne 0\\
		    = (3n_1 - 4m_1) - (3n_2 - 4m_2)\\
		    = 3n_1 - 4m_1 - 3n_2 + 4m_2 \ne 0\\
		    \text{Therefore, it's Injective}\\
		    f(a, b) = c\\
		    3a-4b = c\\
		    3a=c + 4b\\
		    a = (c+4b)/3\\
		    a = c+4b \equiv 0 (mod 3)\\
		    a = 4b \equiv -c (mod 3)\\
		    \text{Therefore, it's Surjective}
	    	\end{split}
	    \end{equation*}
	\item [7]
	    \begin{equation*}
	    	\begin{split}
		    f(0,2) = f(-1,0),\ but\ (0,2) \ne (-1, 0)\\
		    \text{Therefore, it's not Injective}\\
		    f(a, b) = k\\
		    2b-4a = k\\
		    2(b-2a) = k,\ even\\
		    \text{Therefore, it's not Surjective, since if b is odd, the result is not longer even}\\
	    	\end{split}
	    \end{equation*}
	\item [9]
	    \begin{equation*}
	    	\begin{split}
		    f(a) - f(b) \ne 0\\
		    (5a+1)/(a-2) - (5b+1)/(b-2)\\
		    = (5a+1)(b-2) = (5b+1)(a-2)\\
		    = 5ab - 10a + b -2 = 5ab - 10b +a-2\\
		    = - 11a + 11b \ne 0\\
		    \text{Therefore, it's Injective}\\
		    f(a) = b\\
		    (5a+1)/(a-2) = b\\
		    5a+1 = ba - 2b\\
		    5a -ba = -2b -1\\
		    a(5-b) = -2b-1\\
		    a = \dfrac{-2b-1}{5-b}, \in \mathbb{R} - \{5\}\\
		    \text{Therefore, it's Surjective}\\
		    \text{Therefore, it's Bijective}\\
	    	\end{split}
	    \end{equation*}
	\item [15] There are $7^7$ functions. Suppose f is Injective, then there are 7! Injective and Surjective functions. Therefore, there are 7! Bijective functions.
	\item [16]
	\item [17] There are $2^7$ functions. $%TODO: use the notebook notes of cardinality to do these
\end{enumerate}
\begin{enumerate}
	\item 
	\item 
	\item 
	\item 
	\item 
	\item 
	\item 
\end{enumerate}
\section*{Exercises for Section 17.4}
\begin{enumerate}
    \item {(5,1), (6,1), (8,1)}
    \item [3] $g\circ f$ = {(1,1), (2,1), (3,3)}\\
    $f\circ g$ = {(1,1), (2,2), (3,2)}\\
	\item [5] g(f(x)) = $x+1$\\
	    f(g(x)) = $\sqrt[3]{x^3 + 1}$
	\item [6] g(f(x)) = $3(\dfrac{1}{x^2 + 1})+1$\\
	    f(g(x)) = $\dfrac{1}{(3x+2)^2 + 1}$
	\item [7] $g\circ f = (mn + 1,mn + m^2)$\\
    $f\circ g = ((m+1)(m+n), (m+1)^2)$\\
	\item [8] $g\circ f = (5(3m-4n)+2m+n,3m-4n)$\\
    $f\circ g = (3(5m+n)-4m,2(5m+n)+m)$\\
	\item [9] $g\circ f = (m+n,m+n)$\\
    $f\circ g = m+m = 2m$\\
\end{enumerate}
\begin{enumerate}
    \item [i]
	\begin{equation*}
		\begin{split}
			f\circ g \circ h = f(g(h(x)))\\
			= (\dfrac{1}{(x^4)^2 + 1})^3 -4(\dfrac{1}{(x^4)^2 + 1})
		\end{split}
	\end{equation*}
    \item [ii]
	\begin{equation*}
		\begin{split}
			f\circ h \circ g = f(h(g(x)))\\
			= ((\dfrac{1}{x^2 + 1})^4)^3 -4((\dfrac{1}{x^2 + 1})^4)
		\end{split}
	\end{equation*}
    \item [iii]
	\begin{equation*}
		\begin{split}
			h\circ g \circ f = h(g(f(x)))\\
			(\dfrac{1}{(x^3 -4x)^2 + 1})^4
		\end{split}
	\end{equation*}
\end{enumerate}
\section*{Exercises for Section 17.5}
\begin{enumerate}
	\item 
	    \begin{equation*}
	    	\begin{split}
		    \text{Injective}\\
		    f(a) - f(b) \neq 0\\
		    6-a - 6+b = -a+b \neq 0\\
		    \text{Therefore, it's Injective}\\
		    \text{Surjective}\\
		    f(a) = b\\
		    6-a = b\\
		    a = -b+6\\
		    -b+6 \in \mathbb{Z}\\
		    \text{Therefore, it's Surjective}\\
		    \text{Therefore, it's Bijective}\\
		    \text{Inverse}\\
		    m = 6-n\\
		    m-6 = -n\\
		    -m+6 = n\\
		    f^{-1}(n) = -n+6 
	    	\end{split}
	    \end{equation*}
	\item 
	    \begin{equation*}
	    	\begin{split}
		    y = \dfrac{5x+1}{x-2}\\
		    y(x-2) = 5x+1\\
		    yx-2y = 5x+1\\
		    yx-5x = 1+2y\\
		    x(y-5) = 1+2y\\
		    x = \dfrac{1+2y}{y-5}
		    f^{-1}(x) = \dfrac{1+2x}{x-5}
	    	\end{split}
	    \end{equation*}
	\item 
	    \begin{equation*}
	    	\begin{split}
		    \text{Injective}\\
		    f(a) - f(b) \neq 0\\
		    2^a - 2^b \neq 0\\
		    \text{Therefore, it's Injective}\\
		    \text{Surjective}\\
		    f(a) = b\\
		    2^a = b\\
		    a = log_2(b)\\
		    b \in B\\
		    \text{Therefore, it's Surjective}\\
		    \text{Therefore, it's Bijective}\\
		    \text{Inverse}\\
		    f^{-1}(n) = log_2(n) 
	    	\end{split}
	    \end{equation*}
	\item [5]
	    \begin{equation*}
	    	\begin{split}
	    	    y=\pi x - e\\
		    y+e = \pi x\\
		    \dfrac{y+e}{\pi} = x\\
		    f^{-1}(x) = \dfrac{x+e}{\pi}
	    	\end{split}
	    \end{equation*}
\end{enumerate}
\end{document} 
