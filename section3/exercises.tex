\documentclass[12pt]{article}

\usepackage[margin=1cm, paperheight=74in]{geometry}
\usepackage{amsfonts, multicol, pgfplots, tikz}

\pgfplotsset{compat=1.18}
\title{Logic Homework}
\author{Michael Padilla}

\begin{document} 
\maketitle
\section*{Exercises for Section 3.1}
\begin{enumerate}
    \item Every real number is an even integer. \textbf{False}
    \item Every even number is a real number. \textbf{True}
    \item If $x$ and $y$ are real numbers and $5x=5y$, then $x=y$. \textbf{True}
    \item Sets $\mathbb{Z}$ and $\mathbb{N}$. \textbf{Not a statement}
    \item Sets $\mathbb{Z}$ and $\mathbb{N}$ are infinite. \textbf{True}
    \item Some sets are finite. \textbf{True}
    \item[8] $\mathbb{N} \notin P(\mathbb{N})$. \textbf{True}
    \item[11] The integer $x$ is a multiple of $7$. \textbf{Not a statement}
    \item[12] If the integer $x$ is a multiple of $7$, then it is divisible by $7$. \textbf{True}
    \item[13] Either $x$ is a multiple of $7$, or it is not. \textbf{True}
    \item[14] Call me Ishmael. \textbf{Not a statement}
\end{enumerate}
\section*{Exercises for Section 3.2}
\begin{enumerate}
	\item The number 8 is both even and a power of 2.\\
	    p = The number 8 is even\\
	    q = The number 8 is a power of 2\\
	    $p\land q$
	\item The matrix A is not invertible.\\
	    p = matrix A is invertible\\
	    $\neg p$
	\item $x\neq y$\\
	    p = $(x=y)$\\
	    $\neg p$
	\item[5] $y\ge x$\\
	    p = $(y<x)$\\
	    $\neg p$
	\item[7] The number $x$ equals zero, but the number $y$ does not.\\
	    p = The number $x$ equals zero\\
	    q = The number $y$ equals zero\\
	    $p \land \neg q$
	\item[8] At least one of the numbers $x$ and $y$ equals $0$.\\
	    p = The number $x$ equals zero\\
	    q = The number $y$ equals zero\\
	    $p \lor q$
	\item[9] $x\in A-B$\\
	    p = $x\in A$\\
	    q = $x\in B$\\
	    $p \land \neg q$
	\item[10] $x\in A\cup B$\\
	    p = $x\in A$\\
	    q = $x\in B$\\
	    $p \lor q$
	\item[13] Human beings want to be good, but not too good, and not all the time.\\
	    p = Human beings want to be good\\
	    q = Human beings want to be too good\\
	    r = Human beings want to be good all the time\\
	    $p \land \neg q \land \neg r$
	\item[14] A man should look for what is, and not for what he thinks should be.\\
	    p = A man should look for what is\\
	    q = A man should look for what he thinks should be\\
	    $p \land \neg q$
\end{enumerate}
\section*{Exercises for Section 3.3}
\begin{enumerate}
	\item A matrix is invertible provided that its determinant is not zero.\\
	    If a matrix determinant is not zero, then it's invertible.
	\item For a function to be continuous, it is sufficient that it is differenciable.\\
	    If a function is differenciable, then it's continuous.
	\item For a function to be integrable, it is necessary that it is continuous.\\
	    If a function is integrable, then it's continuous.
	\item A function is rational if it is a polynomial\\
	    If a function is a polynomial, then it's rational.
	\item An integer is divisible by 8 only if it is divisible by 4\\
	    If an integer is divisible by 8, then it's divisible by 4.
	\item Whenever a surface has only one side, it is non-orientable\\
	    If a surface has only one side, then it is non-orientable.
	\item A series converges whenever it converges absolutely\\
	    If a series converges absolutely, then it converges.
	\item A geometric series with ratio r converges if $|r| < 1$\\
	    If the ratio r of a geometric series is $|r| < 1$, then it converges.
	\item A function is integrable provided the function is continuous\\
	    If a function is continuous, then it's integrable.
	\item The discriminant is negative only if the quadratic equation has no real solutions.\\
	    If the discriminant of a quadratic equation is negative, then it has no real solutions.
	\item You fail only if you stop writing. (Ray Bradbury)\\
	    If you fail, then you stopped writing.
	\item People will generally accept facts as truth only if the facts agree with what they already believe. (Andy Rooney)\\
	    If people will generally accept facts as truth, then the facts agree with what they already believe.
	\item Whenever people agree with me I feel I must be wrong. (Oscar Wilde)\\
	    If people agree with me, then I feel I must be wrong.
\end{enumerate}
\section*{Exercises for Section 3.4}
\begin{enumerate}
	\item For matrix A to be invertible, it is necessary and sufficient that det(A)$\neq 0$.\\
	    A matrix A is invertible if and only if det(A)$\neq 0$
	\item If a function has a constant derivative then it is linear, and conversely.\\
	    A function has a constant derivative, if and only if it's linear.
	\item If $xy=0$ then $x=0$ or $y=0$, and conversely.\\
	    $xy=0$ if an only if $x=0$ or $y=0$
	\item If $a \in \mathbb{Q}$ then $5a\in\mathbb{Q}$, and if $5a\in\mathbb{Q}$ then $a\in\mathbb{Q}$.\\
	    $a\in \mathbb{Q}$ if and only if $5a\in \mathbb{Q}$
	\item For an occurrence to become an adventure, it is necessary and sufficient for one to recount it.\\
	    An occurrence becomes an adventure if and only if one recounts it.
\end{enumerate}
\section*{Exercises for Section 3.5}
\begin{enumerate}
	\item $P\lor (Q \Rightarrow R)$\\
	    \begin{tabular}{l|l|c|c|c}
		\hline
		P & Q & R & $(Q \Rightarrow R)$ & $P\lor (Q \Leftrightarrow R)$ \\
		\hline
		T & T & T & T & T \\
		T & T & F & F & T \\
		T & F & T & T & T \\
		T & F & F & T & T \\
		F & T & T & T & T \\
		F & T & F & F & F \\
		F & F & T & T & T \\
		F & F & F & T & T \\
		\hline
	    \end{tabular}
	\item $(Q \lor R) \Leftrightarrow (R \land Q)$\\
	    \begin{tabular}{l|l|c|c|c|c}
		\hline
		P & Q & R & $(Q \lor R)$ & $(R \land Q)$ & $(Q \lor R) \Leftrightarrow (R \land Q)$\\
		\hline
		T & T & T & T       & T       & T \\
		T & T & F & T       & F       & F \\
		T & F & T & T       & F       & F \\
		T & F & F & F       & F       & T \\
		F & T & T & T       & T       & T \\
		F & T & F & T       & F       & F \\
		F & F & T & T       & F       & F \\
		F & F & F & F       & F       & T \\
		\hline
	    \end{tabular}
	\item[7] $(P \land \neg P) \Rightarrow Q$\\
	    \begin{tabular}{l|l|c|c|c}
		\hline
		P & Q & $(\neg P)$ & $(P \land \neg P)$ & $(P \neg P) \Rightarrow Q$\\
		\hline
		T & T & F     & F         & T \\
		T & F & F     & F         & T \\
		F & T & T     & F         & T \\
		F & F & T     & F         & T \\
		\hline
	    \end{tabular}
	\item [10] Suppose the statement $((P\land Q) \lor R)\Rightarrow (R \lor S)$ is false. Find the truth values of $P,Q,R,S$\\
	    R = false, S = false, P = true, Q = true
	\item [11] Suppose P is false and that the statement $(R\Rightarrow S) \Leftrightarrow (P \land Q)$ is true. Find the truth values of R and S.\\
	    R = true, S = false
\end{enumerate}
\section*{Exercises for Section 3.6}
\begin{enumerate}
    \item [1] $P \land (Q \lor R) \equiv (P\land Q) \lor (P\land R)$\\
	    \begin{tabular}{l|l|l|c|c|c|c|c}
		\hline
		P & Q & R & $(Q \lor R)$ & $(P \land  Q)$ & $(P \land R)$ & $P \land (Q \lor R)$ & $(P\land Q) \lor (P\land R)$\\
		\hline
		T & T & T & T       & T    & T    & T & T \\
		T & T & F & T       & T    & F    & T & T \\
		T & F & T & T       & F    & T    & T & T \\
		T & F & F & F       & F    & F    & F & F \\
		F & T & T & T       & F    & F    & F & F \\
		F & T & F & T       & F    & F    & F & F \\
		F & F & T & T       & F    & F    & F & F \\
		F & F & F & F       & F    & F    & F & F \\
		\hline
	    \end{tabular}
    \item [3] $P\Rightarrow Q \equiv (\neg P) \lor Q$\\
	    \begin{tabular}{l|l|c|c|c}
		\hline
		P & Q & $(\neg P)$ & $(\neg P \lor Q)$ & $(P\Rightarrow Q)$\\
		\hline
		T & T & F     & T         & T \\
		T & F & F     & F         & F \\
		F & T & T     & T         & T \\
		F & F & T     & T         & T \\
		\hline
	    \end{tabular}
    \item [5] $\neg (P \lor Q \lor R) \equiv \neg P \land \neg Q \land \neg R$\\
	    \begin{tabular}{l|l|l|c|c|c|c|c}
		\hline
		P & Q & R & $(\neg P)$ & $(\neg Q)$ & $(\neg R)$ & $\neg(P \lor Q \lor R)$ & $\neg P \land \neg Q \land \neg R$\\
		\hline
		T & T & T & F & F      & F    & F & F \\
		T & T & F & F   & F    & T    & F & F \\
		T & F & T & F   & T    & F    & F & F \\
		T & F & F & F   & T    & T    & F & F \\
		F & T & T & T   & F    & F    & F & F \\
		F & T & F & T   & F    & T    & F & F \\
		F & F & T & T   & T    & F    & F & F \\
		F & F & F & T   & T    & T    & T & T \\
		\hline
	    \end{tabular}
    \item [7] $P\Rightarrow Q \equiv (P \land \neg Q) \Rightarrow (Q \land \neg Q)$\\
	    \begin{tabular}{l|l|c|c|c|c|c}
		\hline
		P & Q & $(\neg Q)$ & $(P \land \neg Q)$ & $(Q \land \neg Q)$ & $ (P \land \neg Q) \Rightarrow (Q \land \neg Q)$ & $P\Rightarrow Q$\\
		\hline
		T & T & F     & F         & F & T & T \\
		T & F & T     & T         & F & F & F \\
		F & T & F     & F         & F & T & T \\
		F & F & T     & F         & F & T & T \\
		\hline
	    \end{tabular}
	\item [9] $P \land Q$ and $\neg(\neg P \lor \neg Q)$\\
	    $\equiv \neg\neg P \land \neg \neg Q$\\
	    They are logically equivalent by DeMorgan's Law.
	\item [11] $(\neg P) \land (P \Rightarrow Q)$ and $\neg (Q \Rightarrow P)$\\
	    $\neg P \land (\neg P \lor Q) \neq \neg P \land Q$\\
	    They are not logically equivalent.
	\item [13] $P \lor (Q \land R)$ and $(P \lor Q) \land R$\\
	    $(P\lor Q) \land (P\lor R) \neq (R\land P) \lor (R \land Q)$\\
	    They are not logically equivalent.
	\item [A] Prove or disprove: $(P \oplus Q) \oplus R$ and $P \oplus (Q \oplus R)$\\
	    $(P \oplus Q) \oplus R \equiv P \oplus (Q \oplus R)$\\
	    Using the associative laws, we can see they're logically equivalent.
	\item [B] Prove or disprove: $(P \oplus Q) \Rightarrow (P \oplus R)$ and $P \oplus (Q \Rightarrow R)$\\
	    \begin{tabular}{l|l|c|c|c|c|c|c}
		\hline
		P & Q & R & $(P \oplus Q)$ & $(P\oplus R)$ &$ (P \oplus Q) \Rightarrow (P \oplus R)$ & $Q\Rightarrow R$ & $P \oplus (Q \Rightarrow R)$\\
		\hline
		T & T & T & F      & F     & T & T & F \\
		T & T & F & F      & T     & T & F & T \\
		T & F & T & T      & F     & F & T & F \\
		T & F & F & T      & T     & T & T & F \\
		F & T & T & T      & T     & T & T & T \\
		F & T & F & T      & T     & T & F & F \\
		F & F & T & F      & F     & T & T & T \\
		F & F & F & F      & F     & T & T & T \\
		\hline
	    \end{tabular}\\\\
	    They are not logically equivalent, based on the truth table.
\end{enumerate}
\section*{Exercises for Section 7.1}
\begin{enumerate}
    \item $\forall x \in \mathbb{R}, x^2 > 0$\\
	For every Real number x, $x^2$ is positive. \textbf{False}
    \item $\forall x \in \mathbb{R}, \exists n \in \mathbb{N}, x^n \ge 0$\\
	For every Real number x, there's at least one Natural number n, that $x^n$ is zero or positive. \textbf{True}
    \item $\exists a \in \mathbb{R}, \forall x \in \mathbb{R}, ax=x$\\
	There's at least one Real number a, that $ax=x$ for any Real number x. \textbf{True}
    \item $\forall X \in P(\mathbb{N}), X \subseteq \mathbb{R}$ \\
	For every set X in $P(\mathbb{N})$, X is a subset of Real numbers. \textbf{False}
    \item $\forall n \in \mathbb{N}, \exists X \in P(\mathbb{N}), |X| < n$\\
	For every Natural number n, there's at least one subset X of $\mathbb{N}$, that it's cardinatily is less than n. \textbf{True}
    \item $\exists n \in \mathbb{N}, \forall X \in P(\mathbb{N}), |X| < n$\\
    There's at least one Natural number n, that $|x| < n$ for every subset X of $\mathbb{N}$. \textbf{True}
    \item $\forall X \subseteq \mathbb{N}, \exists n \in \mathbb{Z}, |X| = n$\\
	For every subset X of the Natural numbers, there's at least one integer n, that $|X| = n$. \textbf{False}
    \item $\forall n \in \mathbb{Z}, \exists X \subseteq \mathbb{N}, |X| = n$\\
	For every integer n, there's at least one subset X of the Natural numbers, that the cardinatily of X is equal to n. \textbf{True}
    \item $\forall n \in \mathbb{Z}, \exists m \in \mathbb{Z}, m = n + 5$\\
	For every integer n, there's at least one integer m, that $m=n+5$. \textbf{True}
    \item $\exists m \in \mathbb{Z}, \forall n \in \mathbb{Z}, m = n+5$\\
	There's at least one integer m, that $m=n+5$ for every integer n. \textbf{True}
    \item [A] $\exists x \in \mathbb{Z}, \forall y \in \mathbb{Z}, y -x=y$\\
	\textbf{True}
    \item [B] $\forall x \in \mathbb{Z}, \exists y \in \mathbb{Z}, y -x=y$\\
	\textbf{True}
    \item [C] $\exists x \in \mathbb{N}, \forall y \in \mathbb{N}, y-x=y$\\
	\textbf{True}
    \item [D] $\forall x \in \mathbb{N}, \exists y \in \mathbb{N}, y-x=y$\\
	\textbf{False}
    \item [E] $\exists x \in \mathbb{Z}, \forall y \in \mathbb{Z}, y\cdot x = y$\\
	\textbf{True}
    \item [F] $\forall x \in \mathbb{Z}, \exists y \in \mathbb{Z}, y \cdot x =y$\\
	\textbf{True}
    \item [G] $\exists x \in \mathbb{N}, \forall y \in \mathbb{N}, y\cdot x = y$\\
	\textbf{False}
    \item [H] $\forall x \in \mathbb{N}, \exists y \in \mathbb{N}, y\cdot x = y$\\
	\textbf{True}
\end{enumerate}
\section*{Exercises for Section 7.3}
\begin{enumerate}
	\item If $f$ is a polynomial and its degree is greater than 2, then $f'$ is not constant.\\
	    p = f is a polynomial\\
	    q = its degree is greater than 2\\
	    r = $f'$ is constant\\
	    $(p \land q) \Rightarrow \neg r$
	\item The number x is positive, but the number y is not positive.\\
	    p = x is positive\\
	    q = y is positive\\
	    $p \land \neg q$
	\item If x is prime then $\sqrt{x}$ is not a rational number.\\
	    p = x is prime\\
	    q = $\sqrt{x}$ is a rational number\\
	    $p \Rightarrow \neg q$
	\item For every prime number p there is another prime number q with $q>p$\\
	    $\forall p \in primes, \exists q \in primes, q > p$
	\item For every positive number $\varepsilon$, there is a positive number $\delta$ for which $|x-a| < \delta$ implies $|f(x) - f(a)| < \varepsilon$\\
	    $\forall \varepsilon \in \mathbb{R}, \varepsilon > 0, \exists \delta \in \mathbb{R}, \delta > 0, (|x-a| < \delta) \Rightarrow (|f(x) - f(a)| < \varepsilon)$
	\item [7] There exists a real number a for which $a+x=x$ for every real number x.\\
	    $\exists a \in \mathbb{R}, \forall x \in \mathbb{R}, a+x=x$
	\item [8] I don't eat anything that has a face.\\
	    p = I eat some things that have a face\\
	    $\neg p$
	\item [9] If x is a rational number and $x\neq 0$, then tan(x) is not a rational number.\\
	    $((x\in \mathbb{Q}) \land (x \neq 0)) \Rightarrow \tan{x} \notin \mathbb{ Q}$
	\item [10] If sin(x) < 0, then it is not the case that $0\le x \leq \pi$.\\
	    $(sin(x) < 0) \Rightarrow \neg(0\le x \le \pi)$
\end{enumerate}
\section*{Exercises for Section 7.4}
\begin{enumerate}
	\item The number x is positive, but the number y is not positive.\\
	    The number x is not positive, or the number y is positive.
	\item If x is prime, then $\sqrt{x}$ is not a rational number.\\
	    x is prime and $\sqrt{x}$ is a rational number
	\item For every prime number p, there is another prime number q with $q>p$.\\
	    There is at least one prime number p that $q\le p$ for every prime number q.
	\item For every positive number $\varepsilon$, there is a positive number $\delta$ such that $|x-a| < \delta$ implies $|f(x) - f(a)| < \varepsilon$\\
	    There is at least one positive number $\varepsilon$, that $(x-a < d) \land |f(x) - f(a) \ge \varepsilon|$ for every positive number $\delta$
	\item  [6] There exists a real number a for which $a+x=x$ for every real number x.\\
	    For every real number a, there is another real number x such that $a+x\ne x$
	\item  [7] I don't eat anything that has a face.\\
	    I will eat some things that have a face.
	\item  [8] If x is a rational number and $x \ne 0$, then tan(x) is not a rational number.\\
	    X is a rational number and $x\ne 0$ and tan(x) is a rational number.
	\item  [9] If sin(x) $<$ 0, then it is not the case that $0 \le x \le \pi$.\\
	    sin(x) $\ge$ 0 and $0 \le x \le \pi$
	\item  [10] If f is a polynomial and its degree is greater than 2, then $f'$ is not constant.\\
	    F is a polynomial and its degree is greater than 2 and $f'$ is constant.
	    %(p A q) -> ~r
	    %(p A q) A r
	    % x in Q and x not 0 -> tan(x) not in Q
	    % (p A ~q) -> ~r
	    % neg above = (p A ~q) A r
	    %x-a < d -> |fx - fa| < e
	    %x-a >= d or |fx -fa | < e
	    %~(p -> ~q) = p A q
	    %(p -> ~q) =~p V ~q
	    %~(p -> q) = ~p V q
	\item [A] Write $\neg(\exists x \in \mathbb{R}, \forall y \in \mathbb{R}, \exists z \in \mathbb{R}, z > y \Rightarrow z > x^2)$ without using the $\neg$ symbol.\\
	    $\forall x \in \mathbb{R}, \exists y \in \mathbb{R}, \forall z \in \mathbb{R}, (z > y \land z \le x^2)$
	\item [B] Write $\neg(\forall x,y \in \mathbb{R}, x < y \Rightarrow \exists z \in \mathbb{R}, x < z < y)$ without using the $\neg$ symbol.\\
	    $\exists x,y \in \mathbb{R}, x < y \land \forall z \in \mathbb{R}, z \le x \lor z \ge y$
\end{enumerate}
\end{document}
