\documentclass[12pt]{article}

\usepackage[margin=1cm, paperheight=28in]{geometry}
\usepackage{amsfonts, multicol, cancel, amsxtra}

\title{Proof Homework}
\author{Michael Padilla}

\begin{document} 
\maketitle
\section*{Exercises for Section 8}
\begin{enumerate}
    \item $x = 2m, m \in \mathbb{Z}$\\
	$x^2 = 4m^2 = 2(2m^2), n = 2m^2, n \in \mathbb{Z}$\\
	Therefore $x^2$ is a even integer.
    \item [3] $a = 2m + 1, m \in \mathbb{Z}$\\
	$a^2 + 3a + 5 = (2m+1)^2 + 6m + 3 + 5$\\
	$= 4m^2 + 4m + 9 + 6m = 4m^2 + 10m + 9\\
	$= $4m^2 + 10m + 8 + 1 = 2(2m^2 + 5m + 4) + 1$\\
	Where $(2m^2 + 5m + 4) = n \in \mathbb{Z}$, Therefore $a^2 + 3a + 5 = 2n + 1$, and is an odd integer.
    \item [4]
	\begin{equation*}
	    \begin{split}
		x &= 2m + 1, m \in \mathbb{Z} \\
		y &= 2n + 1, n \in \mathbb{Z} \\
		xy &= (2m + 1)(2n + 1)\\
		&= 4mn + 2m + 2n + 1\\
		&= 2(2mn + m + n) + 1\\
		2mn + m +n &= b \in \mathbb{Z}\\
		&= 2b + 1, \text{Therefore it's odd}
	    \end{split}
	\end{equation*}
    \item [5]
	\begin{equation*}
	    \begin{split}
		x &= 2m, m \in \mathbb{Z} \\
		y &= 2n, n \in \mathbb{Z} \\
		xy &= (2m)(2n)\\
		&= 4mn\\
		&= 2(2mn)\\
		2mn &= b \in \mathbb{Z}\\
		&= 2b, \text{Therefore it's even}
	    \end{split}
	\end{equation*}
    \item [7]
	\begin{equation*}
	    \begin{split}
		b = ac, c \in \mathbb{Z}\\
		b^2 = a^2c^2, c^2 = d \in \mathbb{Z}\\
		b^2 = a^2d, \text{Therefore} a^2|b^2
	    \end{split}
	\end{equation*}
    \item [11] 
	\begin{equation*}
	    \begin{split}
		b = am, m \in \mathbb{Z}\\
		d = cn, n \in \mathbb{Z}\\
		bd = (ac)(mn), \text{Therefore} ac|bd
	    \end{split}
	\end{equation*}
    \item [15] 
	\begin{multicols}{2}
	    \begin{equation*}
		\begin{split}
		    \text{Case 1, even n}\\
		    n = 2a, a \in \mathbb{Z}\\
		    n^2 + 3n + 4 = 4a^2 + 6a + 4\\
		    = 2(2a^2 + 3a + 2), (2a^2 + 3a + 2) = c, c \in \mathbb{Z}\\
		    = 2c, \text{Therefore, $n^2 + 3n + 4$ is even}
		\end{split}
	    \end{equation*}
	    \begin{equation*}
		\begin{split}
		    \text{Case 2, odd n}\\
		    n = 2b + 1, b \in \mathbb{Z}\\
		    n^2 + 3n + 4 = 4b^2 + 4b + 1 + 6b + 3 + 4\\
		    = 4b^2 + 10b + 8 = 2(2b^2 + 5b + 4)\\
		    2b^2 + 5b + 4 = d \in \mathbb{Z}, = 2d\\
		    \text{Therefore, $n^2 + 3n + 4$ is even}
		\end{split}
	    \end{equation*}
	\end{multicols}
    \item [16] 
	\begin{multicols}{2}
	\begin{equation*}
	    \begin{split}
		\text{Case 1: Odd parity}\\
		a = 2x+1, x \in \mathbb{Z}\\
		b = 2y+1, y \in \mathbb{Z}\\
		a+b = (2x + 1) + (2y + 1)\\
		= 2x + 2y + 2 = 2(x+y+1)\\
		x+y+1 = z \in \mathbb{Z}, \text{Therefore, their sum is even}
	    \end{split}
	\end{equation*}
	\begin{equation*}
	    \begin{split}
		\text{Case 2: Even parity}\\
		a = 2x, x \in \mathbb{Z}\\
		b = 2y, y \in \mathbb{Z}\\
		a+b = (2x) + (2y)\\
		= 2x + 2y = 2(x+y)\\
		x+y = z \in \mathbb{Z}, \text{Therefore, their sum is even}
	    \end{split}
	\end{equation*}
	\end{multicols}
    \item [17] 
	\begin{equation*}
	    \begin{split}
		a = 2x, x \in \mathbb{Z}\\
		b = 2y + 1, y \in \mathbb{Z}\\
		ab = (2x)(2y + 1)\\
		= 4xy + 2x = 2(2xy + x)\\
		2xy + x = z \in \mathbb{Z}, \text{Therefore, their product is even}
	    \end{split}
	\end{equation*}
\end{enumerate}
\section*{Exercises for Section 9}
\begin{enumerate}
	\item ffff
	\item [3] ffff
	\item [5] ffff
	\item [7] ffff
	\item [9] ffff
	\item [11] ffff
	\item [15] ffff
	\item [17] ffff
	\item [19] ffff
	\item [20] ffff
	\item [23] ffff
\end{enumerate}
\section*{Exercises for Section 10}
\begin{enumerate}
	\item ffff
	\item [3] ffff
	\item [9] ffff
	\item [11] ffff
\end{enumerate}
\begin{itemize}
	\item Prove that the sum of a rational number and an irrational number is always irrational.
	\item Prove that the product of a nonzero rational number and an irrational number is always an irrational number. (Why "nonzero"?)
\end{itemize}
\section*{Exercises for Section 12}
\begin{enumerate}
	\item ffff
	\item [3] ffff
	\item [5] ffff
	\item [9] ffff
	\item [11] ffff
	\item [15] ffff
	\item [17] ffff
	\item [18] ffff
	\item [20] ffff
\end{enumerate}
\section*{Exercises for Section 13}
\begin{enumerate}
	\item ffff
	\item [7] ffff
	\item [9] ffff
	\item [11] ffff
	\item [13] ffff
	\item [29] ffff
\end{enumerate}
\end{document} 
	    \end{split}
	\end{equation*}
	\end{multicols}
    \item [17] 
	\begin{equation*}
	    \begin{split}
	    	ff
	    \end{split}
	\end{equation*}
\end{enumerate}
\section*{Exercises for Section 9}
\begin{enumerate}
	\item ffff
	\item [3] ffff
	\item [5] ffff
	\item [7] ffff
	\item [9] ffff
	\item [11] ffff
	\item [15] ffff
	\item [17] ffff
	\item [19] ffff
	\item [20] ffff
	\item [23] ffff
\end{enumerate}
\section*{Exercises for Section 10}
\begin{enumerate}
	\item ffff
	\item [3] ffff
	\item [9] ffff
	\item [11] ffff
\end{enumerate}
\begin{itemize}
	\item Prove that the sum of a rational number and an irrational number is always irrational.
	\item Prove that the product of a nonzero rational number and an irrational number is always an irrational number. (Why "nonzero"?)
\end{itemize}
\section*{Exercises for Section 12}
\begin{enumerate}
	\item ffff
	\item [3] ffff
	\item [5] ffff
	\item [9] ffff
	\item [11] ffff
	\item [15] ffff
	\item [17] ffff
	\item [18] ffff
	\item [20] ffff
\end{enumerate}
\section*{Exercises for Section 13}
\begin{enumerate}
	\item ffff
	\item [7] ffff
	\item [9] ffff
	\item [11] ffff
	\item [13] ffff
	\item [29] ffff
\end{enumerate}
\end{document} 
