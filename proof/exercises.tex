\documentclass[12pt]{article}

\usepackage[margin=1cm, paperheight=72in]{geometry}
\usepackage{amsfonts, multicol, cancel, amsxtra}

\title{Proof Homework}
\author{Michael Padilla}

\begin{document} 
\maketitle
\section*{Exercises for Section 8}
\begin{enumerate}
    \item $x = 2m, m \in \mathbb{Z}$\\
	$x^2 = 4m^2 = 2(2m^2), n = 2m^2, n \in \mathbb{Z}$\\
	Therefore $x^2$ is a even integer.
    \item [3] $a = 2m + 1, m \in \mathbb{Z}$\\
	$a^2 + 3a + 5 = (2m+1)^2 + 6m + 3 + 5$\\
	$= 4m^2 + 4m + 9 + 6m = 4m^2 + 10m + 9\\
	$= $4m^2 + 10m + 8 + 1 = 2(2m^2 + 5m + 4) + 1$\\
	Where $(2m^2 + 5m + 4) = n \in \mathbb{Z}$, Therefore $a^2 + 3a + 5 = 2n + 1$, and is an odd integer.
    \item [4]
	\begin{equation*}
	    \begin{split}
		x &= 2m + 1, m \in \mathbb{Z} \\
		y &= 2n + 1, n \in \mathbb{Z} \\
		xy &= (2m + 1)(2n + 1)\\
		&= 4mn + 2m + 2n + 1\\
		&= 2(2mn + m + n) + 1\\
		2mn + m +n &= b \in \mathbb{Z}\\
		&= 2b + 1, \text{Therefore it's odd}
	    \end{split}
	\end{equation*}
    \item [5]
	\begin{equation*}
	    \begin{split}
		x &= 2m, m \in \mathbb{Z} \\
		y &= 2n, n \in \mathbb{Z} \\
		xy &= (2m)(2n)\\
		&= 4mn\\
		&= 2(2mn)\\
		2mn &= b \in \mathbb{Z}\\
		&= 2b, \text{Therefore it's even}
	    \end{split}
	\end{equation*}
    \item [7]
	\begin{equation*}
	    \begin{split}
		b = ac, c \in \mathbb{Z}\\
		b^2 = a^2c^2, c^2 = d \in \mathbb{Z}\\
		b^2 = a^2d, \text{Therefore} a^2|b^2
	    \end{split}
	\end{equation*}
    \item [11] 
	\begin{equation*}
	    \begin{split}
		b = am, m \in \mathbb{Z}\\
		d = cn, n \in \mathbb{Z}\\
		bd = (ac)(mn), \text{Therefore} ac|bd
	    \end{split}
	\end{equation*}
    \item [15] 
	\begin{multicols}{2}
	    \begin{equation*}
		\begin{split}
		    \text{Case 1, even n}\\
		    n = 2a, a \in \mathbb{Z}\\
		    n^2 + 3n + 4 = 4a^2 + 6a + 4\\
		    = 2(2a^2 + 3a + 2), (2a^2 + 3a + 2) = c, c \in \mathbb{Z}\\
		    = 2c, \text{Therefore, $n^2 + 3n + 4$ is even}
		\end{split}
	    \end{equation*}
	    \begin{equation*}
		\begin{split}
		    \text{Case 2, odd n}\\
		    n = 2b + 1, b \in \mathbb{Z}\\
		    n^2 + 3n + 4 = 4b^2 + 4b + 1 + 6b + 3 + 4\\
		    = 4b^2 + 10b + 8 = 2(2b^2 + 5b + 4)\\
		    2b^2 + 5b + 4 = d \in \mathbb{Z}, = 2d\\
		    \text{Therefore, $n^2 + 3n + 4$ is even}
		\end{split}
	    \end{equation*}
	\end{multicols}
    \item [16] 
	\begin{multicols}{2}
	\begin{equation*}
	    \begin{split}
		\text{Case 1: Odd parity}\\
		a = 2x+1, x \in \mathbb{Z}\\
		b = 2y+1, y \in \mathbb{Z}\\
		a+b = (2x + 1) + (2y + 1)\\
		= 2x + 2y + 2 = 2(x+y+1)\\
		x+y+1 = z \in \mathbb{Z}, \text{Therefore, their sum is even}
	    \end{split}
	\end{equation*}
	\begin{equation*}
	    \begin{split}
		\text{Case 2: Even parity}\\
		a = 2x, x \in \mathbb{Z}\\
		b = 2y, y \in \mathbb{Z}\\
		a+b = (2x) + (2y)\\
		= 2x + 2y = 2(x+y)\\
		x+y = z \in \mathbb{Z}, \text{Therefore, their sum is even}
	    \end{split}
	\end{equation*}
	\end{multicols}
    \item [17] 
	\begin{equation*}
	    \begin{split}
		a = 2x, x \in \mathbb{Z}\\
		b = 2y + 1, y \in \mathbb{Z}\\
		ab = (2x)(2y + 1)\\
		= 4xy + 2x = 2(2xy + x)\\
		2xy + x = z \in \mathbb{Z}, \text{Therefore, their product is even}
	    \end{split}
	\end{equation*}
\end{enumerate}
\section*{Exercises for Section 9}
\begin{enumerate}
    \item By the contrapositive, suppose If n is odd then $n^2$ is odd
	\begin{equation*}
	    \begin{split}
	    n = 2a + 1, a \in \mathbb{Z}\\
	    n^2 = (2a + 1)^2\\
	    = 4a^2 + 4a + 1 = 2(2a^2 + 2a) + 1\\
	    = 2a^2 + 2a \in \mathbb{Z}\\
	    \text{Therefore, by the contrapositive,If $n^2$ is even, then n is even}
	    \end{split}
	\end{equation*}
	\item [3]
	    \begin{equation*}
	    	\begin{split}
		    a = 2x, x \in \mathbb{Z}\\
		    b = 2y, y \in \mathbb{Z}\\
		    a^2(b^2 - 2b) = 4x^2(4y^2 - 4y)\\
		    = 16x^2y^2 - 16x^2y = 2(8x^2y^2 -8x^2y)\\
		    = 8x^2y^2 -8x^2y \in \mathbb{Z}\\
		    \text{Therefore, by the contrapositive, if $a^2(b^2 -2b)$ is odd, then a and b are odd}
	    	\end{split}
	    \end{equation*}
	\item [5] By the contrapositive, If $x \ge 0$, then $x^2 + 5x \ge 0$
	\item [7]
	    \begin{multicols}{2}
		\begin{equation*}
		\begin{split}
		\text{Case 1: a odd, b even}\\
		    a = 2x + 1, x \in \mathbb{Z}\\
		    b = 2y, y \in \mathbb{Z}\\
		    a\cdot b = 4xy + 2y\\
		    = 2(2xy + y), (2xy + y) \in \mathbb{Z}\\
		    \text{a times b is even}\\
		    a + b = 2x + 2y + 1\\
		    = 2(x + y) + 1\\
		    \text{a + b is odd}
		\end{split}
		\end{equation*}
		\begin{equation*}
		\begin{split}
		\text{Case 2: b odd, a even}\\
		    a = 2x, x \in \mathbb{Z}\\
		    b = 2y + 1, y \in \mathbb{Z}\\
		    a\cdot b = 4xy + 2x\\
		    = 2(2xy + x), (2xy + x) \in \mathbb{Z}\\
		    \text{a times b is even}\\
		    a + b = 2x + 2y + 1\\
		    = 2(x + y) + 1\\
		    \text{a + b is odd}
		\end{split}
		\end{equation*}
	    \end{multicols}
	    \begin{equation*}
	    	\begin{split}
		\text{Case 3: both odd}\\
		    a = 2x + 1, x \in \mathbb{Z}\\
		    b = 2y + 1, y \in \mathbb{Z}\\
		    a\cdot b = 4xy + 2x + 2y + 1\\
		    = 2(2xy + x + y) + 1, (2xy + x + y) \in \mathbb{Z}\\
		    \text{a times b is odd}\\
		    a + b = 2x + 2y + 1 + 1\\
		    = 2(x + y + 1) + 1\\
		    \text{a + b is even}
	    	\end{split}
	    \end{equation*}
	    Therefore, by the contrapositive, in all cases, there's at least one odd when at least a or b are odd
	\item [9]
	    \begin{equation}
	    	\begin{split}
		    n = 3x, x \in \mathbb{Z}\\
		    n^2 = 9x^2 = 3(3x^2)\\
		    \text{Therefore, by the contrapositive, if 3 is not divisible by $n^2$, then 3 is not divisible by n}
	    	\end{split}
	    \end{equation}
	\item [11]
	    \begin{equation*}
	    	\begin{split}
		    x = 2a + 1, a \in \mathbb{Z}\\
		    y = 2b, b \in \mathbb{Z}\\
		    x^2 (y+3) = (4a^2 + 4a + 1)(2b + 3)\\
		    = 8a^2b + 8ab + 2b + 12a^2 + 12a + 2 + 1\\
		    = 2(4a^2b + 4ab + b + 6a^2 + 6a + 1) + 1\\
		    (4a^2b + 4ab + b + 6a^2 + 6a + 1) \in \mathbb{Z}\\
		    \text{Therefore, by the contrapositive, if $x^2(y + 3)$ is even, then x is even or y is odd}
	    	\end{split}
	    \end{equation*}
	\item [15]
	    \begin{equation*}
	    	\begin{split}
		    x = 2a, a \in \mathbb{Z}\\
		    x^3 - 1 = (2a)^3 - 1\\
		    = 8a^3 - 1 = 2(4a^3 - 1) + 1\\
		    4a^3 - 1 \in \mathbb{Z}\\
		    \text{Therefore, by the contrapositive, if $x^3-1$ is even the x is odd}
	    	\end{split}
	    \end{equation*}
	\item [17] 
	    \begin{equation*}
	    	\begin{split}
		    n = 2a + 1, a \in \mathbb{Z}\\
		    n^2 -1 = 8b, b\in \mathbb{Z}\\
		    n^2 - 1 = 4a^2 + 4a = 4a(a + 1)\\
		    a(a+1), \in \mathbb{Z} = 2b\ even = 4(2b) = 8b\\
		    \text{Therefore, by direct proof, if n is odd, then 8 is divisible by $n^2 -1$}
	    	\end{split}
	    \end{equation*}
	\item [19] 
	    \begin{equation*}
	    	\begin{split}
		    a-b = nx, x \in \mathbb{Z}\\
		    a-c = ny, y \in \mathbb{Z}\\
		    c-b = n(x - y)\\
		    \text{Therefore, by direct proof, c$\equiv b$ (mod n)}
	    	\end{split}
	    \end{equation*}
	\item [20] 
	    \begin{equation*}
	    	\begin{split}
		    a - 1 = 5x, x \in \mathbb{Z}\\
		    a = 5x + 1\\
		    a^2 = 25x^2 + 10x + 1\\
		    a^2 - 1 = 25x^2 + 10x\\
		    a^2 - 1 = 5(5x^2 + 2x)\\
		    5x^2 + 2x \in \mathbb{Z}\\
		    \text{Therefore, if a congruence 1 mod 5, then $a^2 \equiv 1$ (mod 5)}
	    	\end{split}
	    \end{equation*}
	\item [23] 
	    \begin{equation*}
	    	\begin{split}
		    a -b = nx, x \in \mathbb{Z}\\
		    ac - bc = nxc, xc = y, y \in \mathbb{Z}\\
		    \text{Therefore, if $a \equiv b (mod\ n),$then $ca\equiv cb (mod\ n)$}
	    	\end{split}
	    \end{equation*}
\end{enumerate}
\section*{Exercises for Section 10}
\begin{enumerate}
	\item 
	    \begin{equation*}
	    	\begin{split}
		    n = 2a + 1, a \in \mathbb{Z}\\
		    n^2 = 4a^2 + 4a + 1 = 2(2a^2 + 2a) + 1\\
		    2b + 1, b = (2a^2 + 2a) \in \mathbb{Z}\\
		    \text{Therefore, by the contradiction, $n^2$ is even and odd, which is a contradiction.}
	    	\end{split}
	    \end{equation*}
	\item [3]
	    \begin{equation*}
	    	\begin{split}
		    \text{By contradiction, $\sqrt[3]{2}$ is rational}\\
		    \sqrt[3]{2} = a/b, \text{In their simpliest form}\\
		    2 = a^3/b^3 = 2b^3 = a^3, a \ is \ even\\
		    2b^3 = (2c)^3 = 8c^3 = 2(4c^3), b\ is \ even \\
		    \text{Therefore, by contradiction, a, b are both even and odd}
	    	\end{split}
	    \end{equation*}
	\item [9] 
	    \begin{equation*}
	    	\begin{split}
		    \text{By contradiction, if a is rational and ab is irrational, then b is rational}\\
		    a = n/m, n,m \in \mathbb{Z }\\
		    b = x/y, x,y \in \mathbb{Z }\\
		    ab = nx/my\\
		    \text{Therefore, by contradiction, ab is both rational and irrational}
	    	\end{split}
	    \end{equation*}
	\item [11] 
	    \begin{equation*}
	    	\begin{split}
		    \text{By contradiction, integers a and b exist, for which 18a+6b = 1}\\
		    1 = 18a + 6b\\
		    1 = 2(9a + 3b)\\
		    \text{Therefore, by contradiction, 1 is even and odd}
	    	\end{split}
	    \end{equation*}
\end{enumerate}
\begin{itemize}
	\item Prove that the sum of a rational number and an irrational number is always irrational.
	    \begin{equation*}
	    	\begin{split}
		    \text{By contradiction, the sum of a rational number and an irrational number is always rational}\\
		    a = x/y, x,y \in \mathbb{Z}\\
		    b = irrational\\
		    a + b = n/m, n, m \in \mathbb{Z}\\
		    b = n/m - x/y\\
		    \text{$n/m\ and\ x/y$, are rational, therefore, by contradiction, b is rational and irrational }
	    	\end{split}
	    \end{equation*}
	\item Prove that the product of a nonzero rational number and an irrational number is always an irrational number. (Why "nonzero"?)
	    \begin{equation*}
	    	\begin{split}
		    \text{By contradiction, a product of a nonzero rational number}\\
		    \text{and an irrational number is always a rational number}\\
		    a = x/y, x, y \in \mathbb{Z}\\
		    ab = n/m, n, m \in \mathbb{Z}\\
		    b = (n/m)/(x/y)\\
		    b = ny/mx\\
		    \text{Therefore, by contradiction, b is both rational and irrational}
	    	\end{split}
	    \end{equation*}
\end{itemize}
\section*{Exercises for Section 12}
\begin{enumerate}
	\item 
	    \begin{equation*}
	    	\begin{split}
		    x = 2a, a \in \mathbb{Z}\\
		    3x + 5 = 6a + 5 = 6a + 4 + 1\\
		    = 2(3a + 2) + 1\\
		    \text{Therefore it's odd}
	    	\end{split}
	    \end{equation*}
	\item [3]
	    \begin{equation*}
	    	\begin{split}
	    	    a = 2n + 1\\
		    a^3 + a^2 + a = (4n^2 + 4n + 1)(2n + 1) + (4n^2 + 4n + 1) + 2n + 1\\
		    = 8n^3 + 8n^2 + 2n + 4n^2 + 4n + 1 + 4n^2 + 4n + 1 + 2n + 1\\
		    = 8n^3 + 16n^2 + 12n + 3 = 2(4n^3 + 8n^2 + 6n + 1) + 1\\
		    a = 2n\\
		    \text{By the contrapositive, it's odd}\\
		    a^3 + a^2 + a = 8n^3 + 4n^2 + 2n = 2(4n^3 + 2n^2 + n)\\
		    \text{Therefore, by direct proof it's even}
	    	\end{split}
	    \end{equation*}
	\item [5] 
	    \begin{equation*}
	    	\begin{split}
	    	    a = 2n + 1\\
		    a^3 = 8n^3 + 12n^2 + 6n + 1\\
		    = 2(4n^3 + 6n^2 + 3n)+1\\
		    \text{It's odd if a is odd}\\
		    a = 2n\\
		    a^3 = 8n^3 = 2(4n^3)\\
		    \text{Therefore, by the contrapositive, a is odd if and only if $a^3$ is odd.}
	    	\end{split}
	    \end{equation*}
	\item [9] 
	    \begin{equation*}
	    	\begin{split}
		    a = 14m, m \in \mathbb{Z}\\
		    a = 7(2m) = 2(7m)\\
		    \text{Therefore, if 14|a then 7|a and 2|a}\\
		    a = 2x, x \in \mathbb{Z}\\
		    a = 7y, y \in \mathbb{Z}\\
		    y = 2z, z \in \mathbb{Z}\\
		    a = 14z\\
		    \text{Therefore, 14 | a if and only if 7|a and 2|a}
	    	\end{split}
	    \end{equation*}
	\item [11] 
	    \begin{equation*}
	    	\begin{split}
		    \text{By the contrapositive, if a is even and b is odd then $(a-3)b^2$ is odd}\\
		    a = 2n, n \in \mathbb{Z}\\
		    b = 2m + 1, m \in \mathbb{Z}\\
		    (a-3)b^2 = (2n - 3)(4m^2 + 4m + 1) = 8nm^2 + 8nm + 2n - 12m^2 - 12m -4 + 1\\
		    = 2(4nm^2 + 4nm + n - 6m^2 - 6m - 2) + 1\\
		    \text{Therefore, it's odd}\\
	    	\end{split}
	    \end{equation*}
	    \begin{multicols}{2}
		\begin{equation*}
			\begin{split}
			    \text{Case 1: a is odd}\\
			    a = 2x + 1, x \in \mathbb{Z}\\
			    (a-3)b^2 = (2x + 1 -3)b^2 = 2(x-1)b^2\\
			    \text{Therefore, it's even}
			\end{split}
		\end{equation*}
		\begin{equation*}
			\begin{split}
			    \text{Case 2: b is even}\\
			    b = 2x, x \in \mathbb{Z}\\
			    (a-3)b^2 = (a-3)(2x)^2 = 2(a-3)2x^2\\
			    \text{Therefore, it's even}
			\end{split}
		\end{equation*}
	    \end{multicols}
	    Therefore, $(a-3)b^2$ is always even
	\item [15] 
	    \begin{equation*}
	    	\begin{split}
		    a = 2m, m \in \mathbb{Z}\\
		    b = 2n + 1, n \in \mathbb{Z}\\
		    a + b = 2m + 2n + 1 = 2(m + n ) + 1\\
		    \text{Therefore, a + b is odd}\\
	    	\end{split}
	    \end{equation*}
	    \begin{multicols}{2}
		\begin{equation*}
			\begin{split}
			    \text{Case 1: a, b are even}\\
			    a = 2m, m \in \mathbb{Z}\\
			    b = 2n, n \in \mathbb{Z}\\
			    a + b = 2m + 2n = 2(m + n)\\
			    \text{It's even}
			\end{split}
		\end{equation*}
		\begin{equation*}
			\begin{split}
			    \text{Case 2: a, b are odd}\\
			    a = 2m + 1, m \in \mathbb{Z}\\
			    b = 2n + 1, n \in \mathbb{Z}\\
			    a + b = 2m + 2n + 2 = 2(m + n + 1)\\
			    \text{It's even}
			\end{split}
		\end{equation*}
	    \end{multicols}
	    Therefore, a + b is even
	\item [17] By the existence method, 97 is that number.
	\item [18] 
	    \begin{equation*}
	    	\begin{split}
		    \text{By the existence method, the set is:} X = \{\mathbb{N}, 1,2,3,...\}
	    	\end{split}
	    \end{equation*}
	\item [20] 
	    \begin{equation*}
	    	\begin{split}
		    \text{By the existence method, the number is: 10}
	    	\end{split}
	    \end{equation*}
\end{enumerate}
\section*{Exercises for Section 13}
\begin{enumerate}
	\item By disproof, when x = -1 and y = -1, $|x+y| = 0, |x| + |y| = 2$
	\item [7] By disproof, when C = $\phi$, A = {1,2,3}, B = {4,5,6}, A$\neq$ B
	\item [9] By disproof, when A = {1,2}, B = {1}, P(A) - P(B) is not a subset of P(A-B)
	\item [11] By disproof, when a = 1, b = 2 $a+b = 3, ab = 2$, so a + b > ab
	\item [13] By existence, the set is $X = \mathbb{R} \cup \{\phi\}$
	\item [29] By disproof, y = 2, x = 0, $|x+y| = |x-y|$, but y is 2
\end{enumerate}
\end{document} 
