\documentclass[12pt]{article}

\usepackage[margin=1cm, paperheight=50in]{geometry}
\usepackage{amsfonts, multicol, amsxtra}

\title{Math Induction Homework}
\author{Michael Padilla}

\begin{document} 
\maketitle
\section*{Exercises for Section 14}
\begin{enumerate}
	\item 
	    \begin{equation*}
	    	\begin{split}
		    \text{Step 1:}\\
		    n = 1, (n^2 + n)/2 = 1, 1 = 1\\
		    \text{Step 2: Suppose}\\
		    k \ge 1, k \in \mathbb{N}\\
		    s(k) = (k^2 + k) / 2\\
		    \text{We want to proof this:}\\
		    s(k + 1) = ((k+1)^2 + k + 1)/2\\
		    (1+2+3+4+...+k)+(k+1) = (k^2 + k)/2 + (k + 1)\\
		    = (k^2 + k + 2k + 1 + 1)/2\\
		    = (k^2 + 2k + 1 + (k + 1))/2\\
		    = ((k+1)^2 + k + 1)/2\\
		    \text{Therefore, we proofed the proposition is true}
	    	\end{split}
	    \end{equation*}
	\item [3] 
	    \begin{equation*}
	    	\begin{split}
		    \text{Step 1:}\\
		    n = 1, (n^2(n+1)^2)/4 = 1, 1 = 1\\
		    \text{Step 2: Suppose}\\
		    k \ge 1, k \in \mathbb{N}\\
		    s(k) = (k^2 (k+1)^2)/4\\
		    \text{We want to proof this:}\\
		    s(k+1) = ((k+1)^2 (k+2)^2)/4\\
		    1^3 + 2^3 + 3^3 + ... + k^3 + (k + 1)^3\\
		    = (k^2 (k+1)^2)/4 + (k + 1)^3\\
		    = (k^2 (k+1)^2 + 4(k + 1)^3)/4\\
		    = ((K+1)^2(k^2+ 4k + 4))/4\\
		    = ((k+1)^2 (k+2)^2)/4\\
		    \text{Therefore, we proofed the proposition is true}
	    	\end{split}
	    \end{equation*}
	\item [4] 
	    \begin{equation*}
	    	\begin{split}
		    \text{Step 1:}\\
		    n = 1, (n(n+1)(n+2))/3 = 2, 2 = 2\\
		    \text{Step 2: Suppose}\\
		    k \ge 1, k \in \mathbb{N}\\
		    s(k) = (k(k+1)(k+2))/3\\
		    \text{We want to proof this:}\\
		    s(k+1) = ((k+1)(k+2)(k+3))/3\\
		    1x2 + 2x3 + 3x4 + ... + k(k+1) + (k+1)(k+2)\\
		    = (k(k+1)(k+2))/3 + (k+1)(k+2)\\
		    = ((k(k+1)(k+2)) + 3(k+1)(k+2))/3\\
		    = ((k+1)(k+2)(k+3))/3\\
		    \text{Therefore, we proofed the proposition is true}
	    	\end{split}
	    \end{equation*}
	\item [5] 
	    \begin{equation*}
	    	\begin{split}
		    \text{Step 1:}\\
		    n = 1, 2^{n+1} -2 = 2, 2=2\\
		    \text{Step 2: Suppose}\\
		    k \ge 1, k \in \mathbb{N}\\
		    s(k) = 2^{k+1} -2\\
		    \text{We want to proof this:}\\
		    s(k+1) = 2^{k+2} -2\\
		    2^1 + 2^2 + 2^3 + ... + 2^k + 2^{k+1}\\
		    = 2^{k+1}-2 + 2^{k+ 1}\\
		    = 2\cdot 2^{k+1} -2 = 2^{k+2} -2\\
		    \text{Therefore, we proofed the proposition is true}
	    	\end{split}
	    \end{equation*}
	\item [6] 
	    \begin{equation*}
	    	\begin{split}
		    \text{Step 1:}\\
		    n = 1, 3 = 4(1) -1, 3 = 3\\
		    \text{Step 2: Suppose}\\
		    k \ge 1, k \in \mathbb{N}\\
		    s(k) = \sum_{i=1}^k(8i-5)=4k^2 -k\\
		    \text{We want to proof this:}\\
		    s(k+1) = \sum_{i=1}^{k+1}(8i-5)=4(k+1)^2 -k -1 = 4k^2 +7k +3\\
		    \sum_{i=1}^{k+1}(8i-5)= \sum_{i=1}^{k} (8i-5) + (8k + 3)\\
		    \sum_{i=1}^{k+1}(8i-5)= 4k^2-k + 8k + 3\\
		    \sum_{i=1}^{k+1}(8i-5)= 4k^2 + 7k + 3\\
		    \text{Therefore, we proofed the proposition is true}
	    	\end{split}
	    \end{equation*}
	\item [7] 
	    \begin{equation*}
	    	\begin{split}
		    \text{Step 1:}\\
		    n = 1, 3 = ((2)(9))/6, 3 = 3\\
		    \text{Step 2: Suppose}\\
		    k \ge 1, k \in \mathbb{N}\\
		    s(k) = (k(k+1)(2k+7))/6\\
		    \text{We want to proof this:}\\
		    s(k+1) = ((k+1)(k+2)(2k + 9))/6\\
		    1x3 + 2x4 + 3x5 + ... + k(k + 2) + (k+1)(k+3)\\
		    = (k(k+1)(2k+7))/6 + (k+1)(k+3)\\
		    = ((k(k+1)(2k+7)) + 6(k+1)(k+3))/6\\
		    = ((k+1)(2k^2+7k + 6k+18))/6\\
		    = ((k+1)(2k^2+13k+18))/6\\
		    = ((k+1)(k+2)(2k+9))/6\\
		    \text{Therefore, we proofed the proposition is true}
	    	\end{split}
	    \end{equation*}
	\item [11] 
	    \begin{equation*}
	    	\begin{split}
		    \text{Step 1:}\\
		    n = 0, 3 | (0 + 0 + 6), 3|6\\
		    \text{Step 2: Suppose}\\
		    k \ge 0, k \in \mathbb{Z}\\
		    3|(k^3 + 5k + 6)\\
		    (k^3 + 5k + 6) = 3x, x \in \mathbb{Z}\\
		    \text{We want to proof this:}\\
		    ((k+1)^3 + 5k + 11) = 3y, y \in \mathbb{Z}\\
		    = 3|(k^2 + 2k + 1)(k+1) +5k +11\\
		    = 3|k^3 + 2k^2 + k + k^2 + 2k + 1 +5k +11\\
		    =3|(k^3 +5k + 6) +3k^2 +  3k +6\\
		    =3|3x +3k^2 +  3k +6\\
		    =3|3(x +k^2 + k +2)\\
		    =(x +k^2 + k +2) = y, y \in \mathbb{Z}\\
		    \text{Therefore, we proofed the proposition is true}
	    	\end{split}
	    \end{equation*}
	\item [13] 
	    \begin{equation*}
	    	\begin{split}
		    \text{Step 1:}\\
		    n = 0, 6|(0) = 0\\
		    \text{Step 2: Suppose}\\
		    k \ge 0, k \in \mathbb{Z}\\
		    k^3 - k = 6x, x \in \mathbb{Z}\\
		    \text{We want to proof this:}\\
		    6|(k+1)^3 - (k + 1)\\
		    = (k^2 + 2k + 1)(k+1) -k -1\\
		    = (k^3 -k) +3k^2 + 3k\\
		    = 6x + 3k^2 + 3k\\
		    = 6x + 3k(k+1)\\
		    \text{Therefore, we proofed the proposition is true, since k(k+1) is even}
	    	\end{split}
	    \end{equation*}
\end{enumerate}
\section*{Additional Questions}
\begin{itemize}
    \item [A] 
	\begin{equation*}
		\begin{split}
		    k^2 + 5k + 1\ even\\
		    p(k+1) = (k+1)^2 + 5(k +1) + 1\ even\\
		    =k^2 + 2k + 1 + 5k +5 + 1 = k^2 + 7k + 7\\
		    = (k^2 + 5k + 1) + 2k + 6\\
		    2k+ 6 = 2(k + 3), \in \mathbb{Z}\ even\\
		    \text{a) Therefore, we proofed the proposition is true}\\
		    \text{Case 1: n is even}\\
		    n = 2m, m \in \mathbb{Z}\\
		    n^2 + 5n + 1 = (2m)^2 + 5(2m) + 1 = 4m^2 + 10m + 1\\
		    = 2(2m^2 + 5m) + 1\ =\ odd\\
		    \text{Case 2: n is odd}\\
		    n = 2m + 1, m \in \mathbb{Z}\\
		    n^2 + 5n + 1 = (2m+1)^2 + 10m + 6\\
		    = 4m^2 + 14m + 2\\
		    = 2(2m^2 + 7m + 2) \in \mathbb{Z} = even\\
		    \text{b) Therefore, when n is even, p(n) is odd and when n is odd, p(n) is even}\\
		    \text{c) Therefore, $p(n) \Rightarrow p(n+1)$ is True, but it doesn't mean that p(n) is always even.}\\
		\end{split}
	\end{equation*}
    \item [B] 
	\begin{equation*}
		\begin{split}
		    \text{Step 1:}\\
		    n = 1, 5 = 3+2\\
		    \text{Step 2: Suppose}\\
		    k > 1, k \in \mathbb{Z}\\
		    f(k) = 3k+2\\
		    \text{We want to proof this:}\\
		    f(k+1) = 3(k+1)+2 = 3k + 5\\
		    f(k+1) = f(k) + 3\\
		    = 3k+5
		    \text{Therefore, we proofed the proposition is true}
		\end{split}
	\end{equation*}
    \item [C] 
	\begin{equation*}
		\begin{split}
		    \text{Step 1:}\\
		    n = 0, 3 = 8-5\\
		    \text{Step 2: Suppose}\\
		    k > 0, k \in \mathbb{Z }\\
		    h(k) = 2^{k+3}-5\\
		    \text{We want to proof this:}\\
		    h(k+1) = 2^{k+4}-5\\
		    h(k+1) = 2h(k) + 5\\
		    = 2(2^{k+3} -5) + 5\\
		    = 2^{k+4} - 5\\
		    \text{Therefore, we proofed the proposition is true}
		\end{split}
	\end{equation*}
    \item [D]
	\begin{equation*}
		\begin{split}
		    \text{Step 1:}\\
		    n = 0, 1 = 2/2\\
		    \text{Step 2: Suppose}\\
		    k > 0, k \in \mathbb{Z}\\
		    g(k) = (3^k + 1)/2\\
		    \text{We want to proof this:}\\
		    g(k+1) = (3^{k+1} + 1)/2\\
		    g(k+1) = 3^{k}+g(k)\\
		    = 3^k + (3^k + 1)/2\\
		    = (2\cdot 3^k + 3^k + 1)/2\\
		    = (3\cdot 3^k + 1)/2\\
		    = (3^{k+1} + 1)/2\\
		    \text{Therefore, we proofed the proposition is true}
		\end{split}
	\end{equation*}
    \item [E]
	\begin{equation*}
		\begin{split}
		    \text{Step 1:}\\
		    n = 1, t(1) = 2, 2=2\\
		    \text{Step 2: Suppose}\\
		    k > 1, k \in \mathbb{Z}\\
		    t(k) = (k+1)!\\
		    \text{We want to proof this:}\\
		    t(k+1) = (k+2)! = (k+1)(k+2)k!
		    t(k+1) = (k+2)t(k)\\
		    = (k+2)(k+1)!\\
		    = (k+2)(k+1)k!\\
		    \text{Therefore, we proofed the proposition is true}
		\end{split}
	\end{equation*}
\end{itemize}
\end{document} 
