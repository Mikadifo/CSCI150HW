\documentclass[12pt]{article}

\usepackage[margin=1cm, paperheight=44in]{geometry}
\usepackage{amsfonts, multicol, pgfplots, tikz}

\pgfplotsset{compat=1.18}
\title{Counting Homework}
\author{Michael Padilla}

\begin{document} 
\maketitle
\section*{Exercises for Section 4.2}
\begin{enumerate}
    \item Consider lists made from the letters T, H, E, O, R, Y, with repetition allowed.
	\begin{itemize}
	    \item Length 4 lists: \textbf{6x6x6x6}
	    \item Length 4 lists that begin with T: \textbf{1x6x6x6}
	    \item Length 4 lists that do not begin with T: \textbf{5x6x6x6}
	\end{itemize}
    \item [3] How many lists of length 3 can be made from the symbols A, B, C, D, E, F if...
	\begin{itemize}
	    \item repetition is allowed: \textbf{6x6x6}
	    \item repetition is not allowed: \textbf{6x5x4}
	    \item repetition is not allowed and the list must contain the letter A:\\
		(A,\_,\_) = 1x5x4\\
		(\_,A,\_) = 5x1x4\\
		(\_,\_,A) = 5x4x1\\
		\textbf{= 3(5x4)}
	    \item repetition is allowed and the list must contain the letter A:\\
		$|U| = 6x6x6, |X^c| = 5x5x5, |X| = (6x6x6)-(5x5x5)$
	\end{itemize}
    \item [5] This problem involves 8-digit binary strings such as 10011011 or 00001010 (i.e.,
8-digit numbers composed of 0’s and 1’s).
	\begin{itemize}
	    \item How many such string are there? \textbf{2x2x2x2x2x2x2x2}
	    \item How many such string end in 0? \textbf{2x2x2x2x2x2x2x1}
	    \item How many such string have 1's for their second and fourth digits? \textbf{2x1x2x1x2x2x2x2}
	    \item How many such string have 1's for their second or fourth digits?\\
		$|A\cup B| = 2x1x2x1x2x2x2x2$\\
		$|A| = 2x1x2x2x2x2x2x2$\\
		$|B| = 2x2x2x1x2x2x2x2$\\
		$= 2^7 + 2^7 - 2^6 = 192$
	\end{itemize}
    \item [7] This problem concerns 4-letter codes made from the letters A, B, C, D, ... , Z.
	\begin{itemize}
	    \item How many such codes can be made? \textbf{26x26x26x26}
	    \item How many such codes have no two consecutive letters the same?\\
		letter 1: any of all 26\\
		letter 2: 26 - the first = 25\\
		letter 3: 26 - the second = 25\\
		letter 4: 26 - the third = 25\\
		\textbf{26x25x25x25}
	\end{itemize}
    \item [9] A new car comes in a choice of five colors, three engine sizes and two transmissions. How many different combinations are there?\\
	Total length is 3, first is 5 colors, second is 3 engine sizes and last is 2 transmissions.\\
	\textbf{5x3x2}
    \item [10] A dice is tossed four times in a row. There are many possible outcomes. How many different outcomes are possible?\\
	Length is 4, a dice has numbers from 1 to 6. \textbf{6x6x6x6}
\end{enumerate}
\section*{Exercises for Section 4.3}
\begin{enumerate}
    \item Five cards are dealt off of a standard 52-card deck and lined up in a row.
	\begin{itemize}
		\item How many such lineups are there that have at least one red card?\\
		    $|U| = $52x51x50x49x48, $|X^c|$ = 26x25x24x23x22\\
		    \textbf{(52x51x50x49x48) - (26x25x24x23x22)}
		\item How many such lineups are there in which the cards are either all black or all hearts?\\
		    They are not black cards that are hearts, so we use the addition principle:\\
		    All black cards: 26x25x24x23x22\\
		    All hearts: 13x12x11x10x9\\
		    \textbf{(26x25x24x23x22) + (13x12x11x10x9)}
	\end{itemize}
    % 3 5 7 9 11
\end{enumerate}
\section*{Exercises for Section 4.4}
\begin{enumerate}
    \item [3] ffff
    % 5 7 9 11 13 15 17
\end{enumerate}
\section*{Exercises for Section 4.5}
\begin{enumerate}
    \item [5] ffff
    % 5 6 7 11 15 17 19
\end{enumerate}
\section*{Exercises for Section 4.7}
\begin{enumerate}
    \item ffff
    % 3 4 (you may skip part c) 7 9 11 13 15
\end{enumerate}
\section*{Exercises for Section 4.8}
\begin{enumerate}
    \item ffff
    % 3 5 7 9 11 13 20
\end{enumerate}
\end{document}
