\documentclass[12pt]{article}

\usepackage[margin=1cm, paperheight=72in]{geometry}
\usepackage{amsfonts, multicol, cancel, amsxtra, graphicx}
\graphicspath{{./imgs/}}

\title{Relation Homework}
\author{Michael Padilla}

\begin{document} 
\maketitle
\section*{Exercises for Section 16.1}
\begin{enumerate}
    \item $R = \{(5,4),(5,3),(5,2),(5,1),(5,0),(4,3),(4,2),(4,1),(4,0),(3,2),(3,1),(3,0),(2,1),(2,0),(1,0)\}$\\
	\includegraphics*[height=6cm]{1}
    \item $R =\{(1,2),(1,3),(1,4),(1,5),(1,6),(1,1),(2,2),(2,4),(2,6),(3,3),(3,6),(4,4),(4,6),(5,5),(6,6)\}$\\
	\includegraphics*[height=6cm]{2}
    \item $R = \{(5,4),(5,3),(5,2),(5,1),(5,0),(4,3),(4,2),(4,1),(4,0),(3,2),$\\
	$(3,1),(3,0),(2,1),(2,0),(1,0), (0,0), (1,1), (2,2), (3,3),(4,4), (5,5)\}$\\
	\includegraphics*[height=6cm]{3}
    \item $A = \{0,1,2,3,4,5\}$,$\\
	$$B=\{(0,0),(0,4),(1,1),(1,3),(2,2),(2,4),(3,3),(3,1),(4,4),(4,0),(4,2),(5,5),(5,1)\}$
    \item $A = \{0,1,2,3,4,5\}$, $B=\{(1,2),(2,5),(3,3),(4,3),(4,2),(5,0)\}$
    \item [9] = $2^{36}$
    \item [11] = $2^{|A|x|A|}$
\end{enumerate}
\section*{Exercises for Section 16.2}
\begin{enumerate}
	\item it is reflexive, symmetric and transitive.
	\item It's not reflexive, becuase $(a, a)$ is missing.\\
	It's not symmetric, becuase $(b, a), (c, a)$ is missing.\\
	It is transitive
	\item It's not reflexive, becuase $(a, a), (b,b), (c,c)$ is missing.\\
	It's not symmetric, becuase $(b, a), (c, a)$ is missing.\\
	It's not transitive, becuase $(b, b), (c, c)$ is missing.\\
	\item it is reflexive, symmetric and transitive.
	\item It's not reflexive, becuase only 0 and $\sqrt{2}$ are in the form of (x,x)\\
	It is symmetric, and transitive\\
	\item [7] R=reflexive, T = transitive and S= symmetric\\
	    \includegraphics*[height=6cm]{4}
	\item [11] It's reflexive, symmetric and transitive 
	\item [12]
	    \begin{equation*}
	    	\begin{split}
		    \text{Reflexive}\\
		    x|x, x=xn, n \in \mathbb{Z}\\
		    x/x = n, n = 1\\
		    x=x(1)\\
		    \text{Therefore, it's reflexive}\\
		    \text{Transitive, if x$|$y and y$|$z, then x$|$z}\\
		    x|y, y = xn, n \mathbb{Z}\\
		    y|z, z = ym, m \mathbb{Z}\\
		    z = xn(m)\\
		    z = x(nm)\\
		    \text{Therefore, it's transitive}
	    	\end{split}
	    \end{equation*}
	\item [13]
	    \begin{equation*}
	    	\begin{split}
		    \text{Reflexive}\\
		    xRx, x-x = 0 \in \mathbb{Z}\\
		    \text{Therefore, it's reflexive}\\ 
		    \text{Symmetric}\\
		    xRy, x -y\in \mathbb{Z}\\
		    = -(x-y) = y-x\\
		    \text{Therefore, it's symmetric}\\ 
		    \text{Transitive}\\
		    xRy \land yRz \Rightarrow xRz\\
		    x-y \in \mathbb{Z }, y-z \in \mathbb{Z}\\
		    (x-y) + (y-z) = x-z \in \mathbb{Z}\\
		    \text{Therefore, it's transitive}
	    	\end{split}
	    \end{equation*}
	\item [15] By counterexample, A = {1,2,3}, R = {(1,1), (1,2), (2,1), (2,2)}, it's not reflexive.
	\item [16]
	    \begin{equation*}
	    	\begin{split}
		    R = \{(x,y) \in \mathbb{Z}x\mathbb{Z} : xRy \Leftrightarrow x^2 \equiv y^2 (mod\ 4)\}\\
		    \text{Reflexive}\\
		    xRx, x^2 \equiv x^2 (mod\ 4), 4|(x^2-x^2)\\
		    4|0, true\\
		    4|0, x^2 -x^2 = (0)\\
		    \text{Therefore, it's reflexive}\\ 
		    \text{Symmetric}\\
		    xRy, x^2 \equiv y^2 (mod\ 4), 4|(x^2-y^2)\\
		    4|(y^2 - x^2), true, yRx\\
		    \text{Therefore, it's symmetric}\\ 
		    \text{Transitive}\\
		    xRy, x^2 \equiv y^2 (mod\ 4), 4|(x^2-y^2), x^2 = y^2 + 4a, a \in \mathbb{Z}\\
		    yRz, y^2 \equiv z^2 (mod\ 4), 4|(y^2-z^2), y^2 = z^2 + 4b, b \in \mathbb{Z}\\
		    x^2 = z^2 + 4a + 4b\\
		    x^2 = z^2 + 4(a+b)\\
		    a+b = c, c \in \mathbb{Z}, \text{c is a multiple of 4}\\
		    4|x^2 -z^2\\
		    \text{Therefore, it's transitive}
	    	\end{split}
	    \end{equation*}
\end{enumerate}
\section*{Exercises for Section 16.3}
\begin{enumerate}
    \item
	\begin{equation*}
	    \begin{split}
		[1] = \{1\}\\
		[2] = \{2, 3\}\\
		[3] = \{3, 2\}\\
		[4] = \{4, 5, 6\}\\
		[5] = \{4, 5, 6\}\\
		[6] = \{4, 5, 6\}\\
	    \end{split}
	\end{equation*}
    \item [3] R = {(a, d),(b, c),(a, a),(c, c),(b, b),(e, e),(d, d),(d, a),(c, b)}
    \item [5] R1 = {(a, a), (b, b), (a, b), (b, a)}, R2 = {(a, a), (b, b)}
    \item [6] R1 = {(a,a),(b,b),(c,c)}\\
	R2 = {(a,a),(b,b),(c,c),(a,b),(b,a),(c,a),(a,c),(c,b),(b,c)}\\
	R3 = {(a,a),(b,b),(c,c), (a,b), (b, a)}\\
	R4 = {(a,a),(b,b),(c,c), (c, a), (a, c)}\\
	R5 = {(a,a),(b,b),(c,c), (b, c), (c, b)}\\
	\item [7]
	    \begin{equation*}
		\begin{split}
		    \text{Reflexive}\\
		    xRx, 3x-5x = -2x, even\\
		    \text{Therefore, it's reflexive}\\
		    \text{Symmetric}\\
		    xRy, 3x-5y = 2a\\
		    3x-5y + 8y -8x = 2a + 8y-8x\\
		    3y -5x = 2(a + 4y -4x)\\
		    \text{Therefore, it's symmetric}\\
		    \text{Transitive}\\
		    xRy, 3x-5y = 2a\\
		    yRz, 3y-5z = 2b\\
		    (3x-5y) +(3y-5z) = 2a + 2b\\
		    3x - 5z = 2a + 2b + 2y\\
		    3x - 5z = 2(a + b + y)\\
		    \text{Therefore, it's transitive}\\
		    \text{Equivalence classes}\\
		    [0] = \{x \in \mathbb{Z}: xR0\} = \{x\in \mathbb{Z}: 3x-0\ even\}\\
		    = \{x\in \mathbb{Z}:x\ even\}\\
		    \text{[0] = All even integers}\\
		    [1] = \{x \in \mathbb{Z}: xR1\} = \{x\in \mathbb{Z}: 3x-5\ even\}\\
		    = \{x\in \mathbb{Z}:x\ odd\}\\
		    \text{[1] = All odd integers}
		\end{split}
	    \end{equation*}
	\item [9]
	    \begin{equation*}
	    	\begin{split}
		    \text{Reflexive}\\
		    xRx, 4|(4x) = 4|4,\ true\\
		    \text{Therefore, it's reflexive}\\
		    \text{Symmetric}\\
		    xRy, 4|(x + 3y), x + 3y = 4n, n\in \mathbb{Z}\\
		    3x + 9y = 12n\\
		    y+3x = 12n - 8y\\
		    y+3x = 4(3n - 2y)\\
		    3n-2y \in \mathbb{Z}, 4|(y+3x)\\
		    \text{Therefore, it's symmetric}\\
		    xRy, 4|(x + 3y), x + 3y = 4n, n\in \mathbb{Z}\\
		    yRz, 4|(y + 3z), y + 3z = 4m, m\in \mathbb{Z}\\
		    x+3y+y+3z = 4n+4m\\
		    x+3z = 4(n+m-y)\\
		    n+m-y \in \mathbb{Z}, 4|(x+3z)\\
		    \text{Transitive}\\
		    \text{Therefore, it's transitive}\\
		    \text{Equivalence classes}\\
		    [0] = \{x \in \mathbb{Z}: 4|x\} = \{\cdots, -4, 0, 4, 8, 12, \cdots\}\\
		    [1] = \{x \in \mathbb{Z}: 4|x+3\} = \{\cdots, -3, 1, 5, 9, 13, \cdots\}\\
		    [2] = \{x \in \mathbb{Z}: 4|x+6\} = \{\cdots, -2, 2, 6, 10, 14, \cdots\}\\
		    [3] = \{x \in \mathbb{Z}: 4|x+9\} = \{\cdots, -1, 3, 7, 11, 15, \cdots\}\\
	    	\end{split}
	    \end{equation*}
	\item [11] This is not true. By the counterexample on the Relation based on Z, where xRy such that 3x-5y is even. It has 2 equivalence classes.
\end{enumerate}
\section*{Exercises for Section 16.4}
\begin{enumerate}
    \item {{a, b}}, {{a}, {b}}
    \item {{a, b, c}}, {{a}, {b}, {c}}, {{a,b},{c}}, {{a}, {b,c}}, {{a, c}, {b}}
    \item 
	\begin{equation*}
		\begin{split}
		    4|(x-0) = \{..., -4, 0, 4,8,...\}\\
		    4|(x-1) = \{..., -3, 1, 5, 9,...\}\\
		    4|(x-2) = \{..., -2, 2, 6, 10, ...\}\\
		    4|(x-3) = \{..., -1, 3, 7, 11, ...\}\\
		    4|(x-4) = \{..., 0, 4, 8, ...\} = [0]\\
		    \text{{[0], [1], [2], [3]}}
		\end{split}
	\end{equation*}
    \item [5]
	\begin{equation*}
		\begin{split}
			xRy, \_ \equiv \_ (mod\ n)\\
			2|(x-0) = \{..., -4, -2, 0, 2, 4,...\}\\
			2|(x-1) = \{..., -5, -3, -1, 1, 3...\}\\
		\end{split}
	\end{equation*}
\end{enumerate}
\section*{Exercises for Section 16.5}
\begin{enumerate}
	\item 
	\item [3]
	\item [4]
	\item [5]
	\item [6]
\end{enumerate}
\end{document} 
